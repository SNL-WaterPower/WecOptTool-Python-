\title{\emph{Design optimization of a WEC}}

\documentclass[]{article}

% ----------------------------------------------------------------------------------

\usepackage{bm}
\usepackage[square,numbers,sort&compress]{natbib}
\bibliographystyle{IEEEtranN}

\usepackage[margin=0.75in]{geometry}
\usepackage{graphicx}
\graphicspath{{./}{./gfx/}}
\usepackage{caption}
\usepackage{subcaption}
\usepackage{amsmath}
\DeclareMathOperator*{\argmin}{arg\,min}
\usepackage[breaklinks,draft]{hyperref}
\def\UrlBreaks{\do\/\do-\do_}
\usepackage{booktabs}
\usepackage{tabu}
\usepackage{multirow}
\usepackage{amssymb}
\usepackage{xcolor}
\usepackage[hang,flushmargin]{footmisc}
\usepackage{ulem}

% \usepackage{draftwatermark}
% \SetWatermarkScale{5}
% \SetWatermarkText{\textbf{DRAFT}}

\usepackage[]{authblk}

% -----------------------------------------------------------

\newcommand{\bigzero}{\mbox{\normalfont\Large\bfseries 0}}
\newcommand{\todo}[1]{\textcolor{red}{\emph{#1}}}

\RequirePackage{snapshot}
\begin{document}

\author[1]{Ryan G. Coe}
\author[1]{Giorgio Bacelli}
\author[1]{Sterling S. Olson}
\author[2]{Mathew B. R. Topper}
\author[1]{Vincent S. Neary}
\author[3]{Alex Hagmuller}
\author[3]{Max Ginsburg}

\affil[1]{Sandia National Laboratories, Albuquerque, NM, USA}
\affil[2]{Data Only Greater}
\affil[3]{AquaHarmonics}

\maketitle

\abstract{
	XX
}

% -----------------------------------------------------------
\section{Introduction}\label{sec:intro}

The methods and formulations here follow those developed by \citet{Bacelli2014}.


% -----------------------------------------------------------
\section{Methods}\label{sec:methods}

% -----------------------------------------------------------
\subsection{Inner optimization problem}\label{sec:inner_optimization_problem}

\begin{figure*}[tb]
	\centering
	\includegraphics[width=0.6\textwidth]{gfx/WecOptTool_structure.pdf}
	\caption{WecOptTool algorithmic and user interface structure. \todo{XX-update diagram with LaTex equations}}
	\label{fig:WecOptTool_structure}
\end{figure*}

We thus wish to solve an optimization problem based on a decision variable $x$, which includes states of the body motion as well as other ``external'' states, e.g., related to controllers, PTO systems, mooring systems, etc.

\begin{equation}\label{eq:decision_variable}
	x =
	\begin{bmatrix}
		x_{pos}\\
		x_{ext}
	\end{bmatrix}
\end{equation}

\noindent{}Here, $x_{pos} \in \mathbb{R}^{m(2n + 1)}$, where $m$ is the number of body modes of motion and $n$ is the number of frequency components.
Similarly, $x_{ext} \in \mathbb{R}^{p(2n+1)}$, where $p$ is the number of external states.
Body modes could include multiple degrees-of-freedom for a single body (e.g., heave, surge, and pitch) and/or degrees-of-freedom from multiple bodies (e.g., heave modes from each body in a large array of bodies).
The frequency vector $\omega \in \mathbb{R}^n$ should span the relevant frequencies of the device operation, and also super harmonics to capture nonlinearities (this topic will be further discussed in Section~\ref{sec:example_tethered_point_absorber}).

\begin{equation}
	x_{pos} =
	\begin{bmatrix}
		x_{pos,1} \\
		x_{pos,2} \\
		\vdots \\
		x_{pos,m}
	\end{bmatrix}
\end{equation}

For each mode, $x_{pos}$ has $2n + 1$ elements ($x_{pos,k} \in \mathbb{R}^{2n + 1}$).
There are $2n$ elements because each frequency is described by two components: one each corresponding to the cosine and sine amplitudes.
The additional element in the first position is used to capture a mean (``DC'') displacement.
Thus, the portion of $x_{pos}$ describing the $k$-th mode would be.

\begin{equation}
	x_{pos,k} =
	\begin{bmatrix}
		\bar{x}_{pos,k} \\
		a_{k,1} \\
		b_{k,1} \\
		\vdots \\
		a_{k,n} \\
		b_{k,n}
	\end{bmatrix}
\end{equation}

\noindent{}The external state vector ($x_{ext}$) has an analogous composition.
The mean components in $x_{ext}$ can be used to capture, for example, pretension in a tether.

The system dynamics is:

\begin{equation}
	(M + m_a) \dot{v} = - k z - B \, v - C \int\limits_{-\infty}^t v  + f_{ext} + f_{exc},
\end{equation}

\noindent{}which can be written in residual for as:

\begin{equation}
	r = -(M + m_a) \dot{v} - k z - B \, v - C \int\limits_{-\infty}^t v  + f_{ext} + f_{exc},
\end{equation}

By defining the quantity $f_i$ as:

\begin{subequations}
	\begin{align}
		f_i &= -(M + m_a) \dot{v} - k z - B \, v - C \int\limits_{-\infty}^t v \\
			&= \mathcal{F}^{-1} \left[ Z \, \hat{V}  \right] ,
	\end{align}
\end{subequations}

\noindent{}where $Z$ is the intrinsic impedance and $\hat{V}$ is the velocity in complex amplitude form, the residual of the system dynamics can be expressed as

\begin{equation}
	r = - f_i + f_{ext} + f_{exc}
\end{equation}

Thus the optimization problem can be written as follows.

\begin{equation}\label{eq:optimization_problem}
	\begin{aligned}
	\min_x p(x)&\\
	\textrm{such that }&\\
	r(x)&=0\\
	c(x)& \leq 0\\
	c_{eq}(x)&=0\\
	\end{aligned}
\end{equation}

\noindent{}The function $p(x)$ gives the total power, where the convention is that power is work done by the WEC, such that negative work is absorbed power, which is desirable.
The dynamics of the system are enforced by the constraint $r(x)=0$.

\begin{equation}\label{eq:residual_form}
	r(x) = \mathcal{F}^{-1}\left\{ Z x_{vel} \right\} - \left( f_{exc}(t) + \sum\limits_{i=1}^p J_i \, f_{ext,i}(t) \right)
\end{equation}

\noindent{}Here, $\mathcal{F}^{-1}$ is the inverse discrete Fourier transform, which is executed by the fast-Fourier transform in practice (i.e., \texttt{ifft}).

In \eqref{eq:residual_form}, $Z\in \mathbb{C}^{m\times{}m\times{}n}$ is the intrinsic impedance of the multi-input, multi-output (MIMO) hydrodynamic system.

\begin{equation}
	Z = B_r + j\left( \omega\left(  M + A_r \right) -\frac{S}{\omega} \right)
\end{equation}

\noindent{}Here, $B_r$ and $A_r$ are the radiation damping and added mass, respectively.
The rigid body inertia is $M$ and the hydrostatic restoring stiffness is $S$.
The radial frequency is $\omega$ and $j=\sqrt{-1}$ is the imaginary unit.

The wave excitation forcing is $f_{exc}(t)$.
This can be expressed in the frequency domain as

\begin{equation}\label{eq:excitation_freq}
	\begin{aligned}
		F_{exc} &= H_{exc}(\omega) \, A(\omega) \\
		&= H_{exc} (\omega) \, e^{j \theta} \sqrt{2 S(\omega) \Delta \omega}.
	\end{aligned}
\end{equation}

\noindent{}Note that \eqref{eq:excitation_freq} gives the complex excitation amplitude ($F_{exc} \in \mathbb{C}^{mn}$), as a random phasing $\theta$ has been introduced.
From $F_{exc}$, one can obtain $f_{exc}(t)$ in the time-domain from the inverse discrete Fourier transform.

\begin{equation}
	f_{exc}(t) = \mathcal{F}^{-1} \left[ F_{exc}(\omega) \right]
\end{equation}

External forces and moments, such as those due to a PTO or a mooring system, act on the free hydrodynamic modes according to a kinematic Jacobian $J_i$.
In general, the periodic time-history of each external force should be determined based on the decision variable $x$ and can take any form.

% \begin{equation}
% 	f_{exc}(t) = f(x)
% \end{equation}

The velocity $x_{vel}$ in \eqref{eq:residual_form} is obtained from

\begin{equation}
	x_{vel} = D_{\phi} \, x_{pos},
\end{equation}

\noindent{}where $D_{\phi} \in \mathbb{R}^{m(2n+1) \times m(2n+1)}$ is the differentiation matrix, which is a block diagonal matrix.

\begin{equation}\label{eq:Dphi}
	D_{\phi} =
	\begin{bmatrix}
	D_{\phi,1} & & \\
	& \ddots & \\
	 & & D_{\phi,m}
	\end{bmatrix}
\end{equation}

\noindent{}Each block within $D_{\phi}$ corresponds to a mode of motion.
For the specific case of Fourier expansion, these are defined by combining a matrix with the frequency components along the upper and lower diagonals

\begin{equation}\label{eq:Dphi_blocks}
D^d_{\phi,k} = \begin{bmatrix}
	\begin{matrix} 0 &  \omega_0 \\ -\omega_0 & 0 \end{matrix} \\
	& \begin{matrix} 	0 & 2 \, \omega_0 \\ -2 \, \omega_0 & 0	\end{matrix} \\
	\phantom{\begin{matrix} 	0 & 2 \, \omega_0 \\ -2 \, \omega_0 & 0	\end{matrix}}\\
	&&&&\begin{matrix} 0 & n \, \omega_0 \\ -n \, \omega_0 & 0 \end{matrix} \\
\end{bmatrix}
\end{equation}

\noindent{}with a column vector of zeros

\begin{equation}
	D_{\phi,k} = [0_{2n\times1},\, D^d_{\phi,k}] .
\end{equation}

\noindent{}Thus, we have for each mode $k=1,2,\dots m$ a matrix $D_{\phi,k} \in \mathbb{R}^{2n\times2n+1}$.
Combining the block matrix for each mode according to \eqref{eq:Dphi}, we arrive at $D_{\phi} \in \mathbb{R}^{2mn \times m(2n+1)}$.

The Fourier coefficient vector $x_{pos}$ is built as
\begin{equation}
	x_{pos} = [a_1, b_1, a_2, b_2, \ldots, a_n, b_n],
\end{equation}
thus the resulting coefficients for the velocity vector are:
\begin{equation}
	\begin{aligned}
		x_{vel} &= D_{\phi} \, x_{pos}\\
		&= [\omega_0 \, b_1, -\omega_0 \, a_1, \ldots, n \, \omega_0 b_n, -n \, \omega_0 \, a_n].
	\end{aligned}
\end{equation}

\noindent{}From the velocity vector, which contains these cosine and sine components for each frequency, we can find a complex velocity vector.
The k-\textit{th} element of the complex amplitude vector for the velocity $\hat{V}$ is built as

\begin{equation}\label{eq:ps_to_ca}
	\hat{V}_k = x_{vel_{2k-1}} - i \, x_{vel_{2k}}, \qquad \textrm{for } k=1:n.
\end{equation}

\noindent{}Finally, we may construct the velocity's periodic time history at the collocation points as

\begin{equation}
	v(t) = \sum\limits_i a_i \cos \left( k \omega_0 t \right) + b_i \sin \left( k \omega_0 t \right) .
\end{equation}


The additional constraints in \eqref{eq:optimization_problem} can be used to limit body motion or PTO inputs (i.e., through $c(x)$) and obtain a non-zero mean, e.g., in the case of a tethered device (i.e., through $c_{eq}(x)$).


The impedance has to be reshaped into a block matrix $Z_{block} \in \mathbb{C}^{mn \times mn}$.
Each block within $Z_{block}$ is a diagonal matrix $Z_{block,ik} \in \mathbb{C}^{n \times n}$ corresponding to the interaction between the $i$-th and $k$-th modes of motion.

\begin{equation}
	Z_{block,ik} =
	\begin{bmatrix}
		Z_{ik}(\omega_0) & & \\
		& \ddots & \\
		& &  Z_{ik}(n \omega_0)
	\end{bmatrix}
\end{equation}

% -----------------------------------------------------------------------------------------
\subsection{Numerical implementation}\label{sec:numericalImplementation}
In practice, the number of decision variables will often be large.
The decision variable $x$, as specified by \eqref{eq:decision_variable}, with have a size of $(2n + 1) \cdot (m + p)$, where $n$, $m$, and $p$ are the number of frequency components, the number of body modes of motion, and the number of external states, respectively.
The number of frequency components needed will depend first on the dynamics of the device and the wave spectrum.
The degree to which either the device or spectrum exhibit narrow banded or ``peaky'' responses will increase the need for larger numbers of frequency components.
The National Buoy Data Center (NDBC), for instance, currently uses 47 frequency components between 0.02 and 0.485\,Hz in spectral density estimates.
Considering next that we would like to handle nonlinearities which will occur at super-harmonics of the exciting frequencies, we need five times the number of these exciting frequencies.

Take, for example, a single degree of freedom WEC operating in an irregular sea state.
If we use $n=25$ frequency components, which is fairly parsimonious, the result would be $(2\cdot 5 \cdot 25 + 1) \cdot (1 + p) > 251$ decision variables.
Of course $p \geq 1$, so it is clear to see how the number of decision variables can easily reach $\mathcal{O}(1e3)$.

The optimization problem described in Section~\ref{sec:inner_optimization_problem} requires a high-accuracy deterministic solution.
We cannot, in general, ensure that the Jacobians of the objective and constraint functions will be known.
Given these factors along with the understanding that the number of decision variables will often be large, the problem is well-suited to the application of automatic differentiation \todo{XX-cite}.

% -----------------------------------------------------------------------------------------
\subsubsection{Scaling}\label{sec:scaling}

Elementwise multiplication of $x$ to scale.
User sets this based on knowledge of problem (for now).
Pre-multiply $x$ in \texttt{decompose\_decision\_variable}.
Write Obj fun and PTO fun and external fun as if working with non-scaled $x$.
Final step after solver runs, unscale results.

\section{Example: tethered point absorber}\label{sec:example_tethered_point_absorber}

To better illustrate the concepts outlined in Section~\ref{sec:methods}, we consider an example for a tethered point absorber.

\bibliography{WecOptTool_FY21Q2.bib}   % name your BibTeX data base

\end{document}
